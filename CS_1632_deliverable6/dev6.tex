\documentclass[titlepage]{article}
\title{Property Based Testing with Haskell and QuickCheck\\
CS 1632 - FINAL DELIVERABLE\\
\small{https://github.com/blester125/CS\_1632\_deliverable6}}
\author{Brian Lester}
\date{19 April 2016}
\begin{document}
\maketitle
\section{Summary}
During Deliverable 4 I did property based testing for the Arrays.sort() method in 
Java. This was pretty fun and interesting. One slight annoyance during this 
project was the fact we had to create the random input ourselves. This was easy 
but it seemed like something that could be done for us. Luckily Haskell has this 
in the form of QuickCheck. I have been learning some Haskell in my free time so 
when the final deliverable was announced to be free form it seemed like a perfect 
opportunity to learn more about Haskell and to test out the QuickCheck library.

In this deliverable I created some simple QuickChecks of various mathematical 
properties. After that I create some tests for various sorting methods I have 
implemented. Finally I created more complex tests that test a Data Structure 
(Binary Search Tree) that I implemented. All tests passed by the end of 
development.

This deliverable was a lot of fun, using some online resources I was able to 
expand my knowledge of Haskell in general (I had never implemented a BST before) 
and specifically my knowledge of testing with QuickCheck.

The only part about this that I would have concerns about is the fact that I was 
not able to apply these tests to an actual application. This is mostly do to my 
inexperience with Haskell. An idea I had was to implement Huffman Encoding in 
Haskell but I was having trouble applying these tests to the actual Huffman 
encoding (which I did get working). The only thing I could test was the BST used 
to back the Huffman tree. I would have liked to figure out how to do this but I 
unfortunately have to much other course work and am too inexperienced with 
Haskell. However I did have a lot of fun.

The quality of the Haskell Code I wrote is top notch. There are no failed tests 
and all the functionality works. The code is ready to be released as a usable 
Binary Search Tree that could be used to back a set or something similar. The 
Tree implements the BST invariant and the insertion and deletions maintain this 
invariant. The tests of the tree prove it's correctness which means that it can 
be released. The Sorts are also tested and fact that they have the properties of 
sorts (plus they match the library sort) show that they are correct and can be 
used to sort things in different conditions.

The code is available at https://github.com/blester125/CS\_1632\_deliverable6

\section{Getting Started with QuickCheck}
\subsection{The First Test}
The first step was obvious to test out QuickCheck. I started with a simple test 
that tested the properties of reversing lists. This properties is that if you 
append one list to another and then reverse the whole thing then it is the same 
as reversing each list and appending the first list to the second one. It would 
look like this: 

\begin{centering}
	x = \lbrack 1,2,3\rbrack \\
	y = \lbrack 4,5,6\rbrack \\
	reverse (x + y) = reverse (\lbrack 1,2,3,4,5,6\rbrack ) \\
	= \lbrack 6,5,4,3,2,1\rbrack \\
	and \\
	x = \lbrack 1,2,3\rbrack  \\
	y = \lbrack 4,5,6\rbrack \\
	reverse(y) + reverse(x) \\
	\lbrack6,5,4\rbrack + \lbrack 3,2,1\rbrack \\
	  = \lbrack6,5,4,3,2,1\rbrack \\
\end{centering}
This example is used in several tutorials for QuickCheck so this seemed like a 
good starting point. I also include a few other tests to get acclimated to 
QuickCheck such as the commutative property of addition and the associativity of 
multiplication and addition.

This code can be found in ``Example.hs''. To run this code load it into GHCI with 
``:l Example.hs'' and type main. 

\subsection{QuickCheck and Sorting}
Sorting introduced me to property based testing (it makes sense, there are a lot 
of properties to test with sorting) so it makes sense that sorting would be the 
next logical step to exploring QuickCheck. For this part I tested the following 
properties of the sort methods:
\begin{enumerate}
	\item Idempotent property of the sort
	\item Increasing ordering of the final list
	\item The minimum of the list is at the head of the list
	\item The maximum item is at the end of the list
	\item The sorted List is a permutation of the original list (this property 
	covers the property that the number of elements in the input are equal to the 
	number in the output, the property that there are no elements in the input 
	that are not in the output and visa versa)
	\item The smallest item in a sorted list composed of appending two lists is 
	      the smallest item in both lists.
	\item Testing against a model in the form of the build in sort
\end{enumerate}
I wrote several sort methods to test, quickSort (a Haskell Classic), mergeSort, 
and BubbleSort.

When writing these test the first hurdle I ran into was in the minimum tests. 
minimum is not defined for empty lists so I had to learn to write a test that 
will not generate a null test. This was also important in the appended list tests 
because the use of ``head'' in the tests

The next hurdle was testing three different sorting functions. I know about 
higher order functions and currying so I first tried to write one test that I 
could pass both a list and a sort function to use. I was unable to figure out 
how to get this to work with QuickCheck. I had to resort to writing different 3 
different tests.

This code can be found in ``Sorts.hs''. To run this code load it into GHCI with ``
:l Sorts.hs'' and type main.

\section{More Advanced Checks - Binary Search trees}
While the tests for the sort were quite long and I tested more properties than I 
did in deliverable 4 these tests were not too much more in depth than the tests 
from that deliverable. To expand on this idea I decided to do property based 
testing on a Data Structure I implemented. I chose to implement a Binary Search 
Tree because it has a nice invariant to test. This is the binary search tree 
property which states that all the entries in the left sub tree is smaller than 
the entries in the right subtree. 

These tester were much harder to write than the sort tests. All the sort test 
were easy because all QuickCheck had to was generate random numbers. In this case 
we needed to generate correct Binary Search Trees otherwise things like the 
Insert test could fail, not because the insert failed but rather because the 
Binary Search Tree was not valid to begin with.

One problem with these tests is the delete tests. QuickCheck generates random 
numbers to delete from the tree. Not all of these are necessarily in the tree so 
the test is on average likely to be less thorough as I would like it to be.  

This code can be found in ``BST.hs''. To run this code load it into GHCI with ``
:l BST.hs'' and type main.
\end{document}