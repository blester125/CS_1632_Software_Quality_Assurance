\documentclass[titlepage]{article}
%Compile with pdflatex%
\title{CS 1632 -- DELIVERABLE 5\\
Performance Testing Conway's Game of Life\\
Project under Test: JavaLife\\
\small{\url{https://github.com/blester125/SlowLifeGUI}}}
\author{Brian Lester}
\usepackage{hyperref}
\usepackage{graphicx}
\usepackage{listings}
\usepackage{amsmath}
\newcommand{\testcase}[7]{IDENTIFIER: {#1}\\
TEST CASE: {#2}\\
PRECONDITIONS: {#3}\\
INPUT VALUES: {#4}\\
EXECUTION STEPS: {#5}\\
OUTPUT VALUES: {#6}\\
POSTCONDITIONS: {#7}\\
\\
}
\begin{document}
\maketitle
\section{Summary}
For deliverable 5 I profiled the Game of Life provided with the Visual VM 
profiler. In the deliverable description, the code was described as non 
performent. This meant that the CPU time for each method was most likely a good 
metric for finding methods that were causing problems. By running the Game of 
Live, Visual VM, and using the CPU sampler I looked for ``Hot Spots''. This 
identified the method convertToInt(int x) in the MainPanel class was a 
``Hot Spot''. This is where I began my investigation.

This method was the highest on the profiler list so I investigated it. Once I 
looked at it I saw that the method was converting ``int''s to ``int''s which 
means that it probably doing extra stuff because the entire method seem pretty 
pointless. Upon closer inspection I found that function was creating a large 
string that it would concatenate the ``int x'' onto then call the Integer.
parseInt() method on it. This is clearly extra and pointless operations so I 
decided that I could refactor this code to fix the problem.

Using exploratory testing I found that the method acted as follows. 
\[
	\begin{cases}
		NumberFormatException & x < 0\\
		x & x \geq 0\\
	\end{cases}
\]
Once I figured out how the method worked I wrote the required pinning test to 
show that I did not change the functionality of the program. First I changed the 
method declaration from private to public to make testing easier. I first wrote 
test 
for the three equivalence class, negative numbers, zero, and positive numbers. 
For the negative number equivalence class I created a test to see if it still 
throw the ``NumberFormatException'' because the pinning tests are for current 
behavior not expected behavior. Finally I wrote a property based test for the 
method. The property that I tested was that when a number greater than 0 is 
passed in it returns the same number. I tested this by randomly testing the 
method with 100 integers between 0 and 400. The pinning tests can be found in the 
``MainPanelTest.java'' file.

After fixing the ``converToInt()'' method I ran the profiler again and examined 
the profile when I write to the file. This identified the ``Cell.toString''
method as a ``Hot Spot''. When I was exploratory testing I had 
noticed that the file output was really slow so this makes sense as a 
``Hot Spot''. 

After examination I found that the method worked like so.
\[  
	\begin{cases}
		\text{``X''} & \text{text starts with X} \\ 
		\text{``.''} & \text{text doesn't start with X} \\
	\end{cases}
\] 
I wrote pinning tests to confirm this behavior after the refactor. The pinning 
tests can be found in the ``CellTest.java'' file.

Running the profiler again I found another hotspot. The profiler identified Thread.sleep() as the hotspot but this seemed to used for user input so I kept looking. The first method that I could reasonably change (The first method that was not a jswing function) was the MainPanel.runContinuous() function. Investigating I found a loop that ran a bunch and changed the ``\_r'' value. However, before the loop the value of ``\_r'' is saved into a variable ``origR'' and after the loop the value of ``\_r'' is overwritten by ``origR''. This means that the value of ``\_r'' is not changed by the loop so it could be taken out. Pinning tests for this method are manual and are included in this document.

The included screen shots show that the refactored methods are no longer a ``Hot 
Spots'' and the JUint screen shots show that the behavior of the methods were not 
changed.

\newpage
\section{Before Refactoring of MainPanel.convertToInt()}
\subsection{Profiler ScreenShot}
\includegraphics[width=\textwidth, natwidth=975,natheight=425]{hotSpotMethod.png}
\subsection{Pinning Tests}
\includegraphics[width=.75\textwidth, natwidth=400,natheight=640]{Old_Tests.png}
\section{After Refactoring of MainPanel.convertToInt()}
\subsection{Profiler ScreenShot}
\includegraphics[width=\textwidth, natwidth=978,natheight=296]{newhotspot.png}
\subsection{Pinning Tests}
\includegraphics[width=.75\textwidth, natwidth=400,natheight=640]{New_Tests.png}

\section{Before Refactoring of Cell.toString()}
\subsection{Profiler ScreenShot}
\includegraphics[width=\textwidth, natwidth=978,natheight=425]{HotSpot2.png}
\subsection{Pinning Tests}
\includegraphics[width=.75\textwidth, natwidth=400,natheight=640]{Old_toStringTests.png}
\section{After Refactoring of Cell.toString()}
\subsection{Profiler ScreenShot}
\includegraphics[width=\textwidth, natwidth=978,natheight=275]{newHotSpot2.png}
\subsection{Pinning Tests}
\includegraphics[width=.75\textwidth, natwidth=400,natheight=640]{New_StringTests.png}

\section{Before Refactoring of MainPanel.runCountinous()}
\subsection{Profiler ScreenShot}
\includegraphics[width=\textwidth, natwidth=727, natheight=230]{3rdHotSpot.png}
\subsection{Pinning Tests}
This method would be very difficult to test with automatic tests so we will need to use manual testing.\\
\\
\testcase
{1. FUN-BUTTON-STARTS-THE-GAME}
{Testing that pressing the run continuous button runs the game}
{\\
 The Board is 8x8.\\
 The cells at row 0 column 1, row 1 column 2, row 2 column 0,1,2 are alive.}
{N/A}
{Click the ``Run Continuous'' button}
{N/A}
{The game should begin and cells should start changing}
\testcase
{2. FUN-MOVE-2-ITERATIONS}
{Move multiple iterations}
{\\
 The Board is 8x8.\\
 The cells at row 0 column 1, row 1 column 2, row 2 column 0,1,2 are alive.}
{N/A}
{Press the ``Run Continuous'' button.\\
Let the game run for  iterations.\\
Press the ``Stop'' button.}
{N/A}
{The cells at row 1 column 2, row 2 column 0 and 2, row 3 column 1 and 2 are alive.}
\testcase
{3. FUN-UNDO-AFTER-RUN}
{Try to undo after running for a bit.}
{\\
 The Board is 8x8.\\
 The cells at row 0 column 1, row 1 column 2, row 2 column 0,1,2 are alive.}
{N/A}
{Press the ``Run Continuous'' button.\\
Let the game run for  iterations.\\
Press the ``Stop'' button.\\
Press the ``Undo'' button.}
{N/A}
{The cells at row 1 column 0 and 2, row 2 column 1 and 2, row 3 column 1 are alive.}

\section{After Refactoring of MainPanel.runContinous()}
\subsection{Profiler ScreenShot}
\includegraphics[width=\textwidth, natwidth=978, natheight=212]{3rdHotSpotFixed.png}
\subsection{Pinning Tests}
All the manual test should still pass.
\end{document}
